\documentclass[11pt]{article} 

\usepackage{xcolor}
\usepackage{titlesec} 
\usepackage{graphicx} 
\usepackage{pgfplots}
\usepackage{pgfplotstable} 
\pgfplotsset{compat=1.17} 
\usepackage{tikz}
\usepackage{hyperref}
\usepackage{pdfcomment}

\definecolor{section}{RGB}{0, 102, 153}
\definecolor{sub_section}{RGB}{128, 0, 32}
\definecolor{titi}{RGB}{0, 128, 64}

\titleformat{\section}
  {\normalfont\fontsize{17}{17}\bfseries}{\thesection}{1em}{}
  
\title{Document Green IA}   
\author{PICHARD Quentin, GUEYE Massamba, CHARLEMAGNE Clément}               
\date{\today}             

\begin{document}

\maketitle

\begin{center}
    \includegraphics[width=\linewidth]{/home/charlemagne/Images/illustrations/developpement_durable/pexels-lara-jameson-9324352.png}
\end{center}

\section{\textcolor{section}{Présentation du projet\\}}
\subsection*{\textcolor{sub_section}{Idées en vrac:}}
Application web dispo sur mobile et ordinateur (responsive, HTML, CSS, JS),
création de données grace à des scripts, puis ajout de données réelles à l'
échelle d'un département ou d'une circonsription française. Algorithme de recommandation 
de plats en fonction de ses habitudes, contraintes alimentaires (peut etre dans une v2 les 
contraintes alimentaires) et de son département et de la saison en cours. Barre de recherche 
intelligente, permettant à l'utilisateur d'entrer une phrase ou une liste de mots, qui 
permettront de lancer une recherche précise (comme la SNCF). 

\subsection*{\textcolor{sub_section}{Présentation du projet:}}

\subsection*{\textcolor{sub_section}{Fonctionnalités à développer:}}
\begin{itemize}
  \item Scanner code barre d'un article.  
  \item Afficher les consignes de recyclages.  
  \item Afficher l'impact carbone.
  \item Donner des alternatives moins émettrices de carbone, dans la base de données 
  \href[pdfnewwindow=true]{https://fr.openfoodfacts.org/}{OpenFoodFacts} (algo de
  recommandation). 
  \item Ajouter les dates de collecte des déchets autour de chez sois. 
  \item Ajouter des points de collecte (hors carte). 
  \item Barre de recherche intelligente qui récupère les informations
  intéressantes de sa phrase, pour lui afficher les jours et zones de ramassage, 
  des informations carbones sur un article, des commerces proches de chez lui qui 
  pourraient correspondrent à ces attentes (chaque utilisateur pouvant sélectionner 
  des mots clés proposés par l'application à la suite d'une visite chez un commerçant, 
  des articles qu'il y a trouvé ou non).
\end{itemize}

\section{\textcolor{section}{Gestion de projet\\}}
\subsection*{\textcolor{sub_section}{Liste des taches:}}
\subsection*{\textcolor{sub_section}{Répartition des taches:}}
\subsection*{\textcolor{sub_section}{KPI et RIO:}}
\subsection*{\textcolor{sub_section}{Analyse des risques:}}
\subsection*{\textcolor{sub_section}{KPI et RIO:}}
\subsection*{\textcolor{sub_section}{Graphique d'avancement:}}
\subsection*{\textcolor{sub_section}{Avancées individuelles:}}

\end{document}
